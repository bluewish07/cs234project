%%%%%%%% ICML 2018 EXAMPLE LATEX SUBMISSION FILE %%%%%%%%%%%%%%%%%

\documentclass{article}

% Recommended, but optional, packages for figures and better typesetting:
\usepackage{microtype}
\usepackage{graphicx}
\usepackage{subfigure}
\usepackage{booktabs} % for professional tables

% hyperref makes hyperlinks in the resulting PDF.
% If your build breaks (sometimes temporarily if a hyperlink spans a page)
% please comment out the following usepackage line and replace
% \usepackage{icml2018} with \usepackage[nohyperref]{icml2018} above.
\usepackage{hyperref}

% Attempt to make hyperref and algorithmic work together better:
\newcommand{\theHalgorithm}{\arabic{algorithm}}

% Use the following line for the initial blind version submitted for review:
%\usepackage{icml2018}

% If accepted, instead use the following line for the camera-ready submission:
\usepackage[accepted]{icml2018}

% other packages
\usepackage{verbatim}
\usepackage{url}
\usepackage{amsmath,amssymb}
\usepackage{makecell}
\usepackage{enumitem}

\graphicspath{ {images/} }

% The \icmltitle you define below is probably too long as a header.
% Therefore, a short form for the running title is supplied here:
\icmltitlerunning{Generalized Multi-Agent Reinforcement Learning}

\begin{document}

\twocolumn[
\icmltitle{Generalized Multi-Agent Reinforcement Learning in \\ 
	       Cooperative and Competitive Environments}

% It is OKAY to include author information, even for blind
% submissions: the style file will automatically remove it for you
% unless you've provided the [accepted] option to the icml2018
% package.

% List of affiliations: The first argument should be a (short)
% identifier you will use later to specify author affiliations
% Academic affiliations should list Department, University, City, Region, Country
% Industry affiliations should list Company, City, Region, Country

% You can specify symbols, otherwise they are numbered in order.
% Ideally, you should not use this facility. Affiliations will be numbered
% in order of appearance and this is the preferred way.
\icmlsetsymbol{equal}{*}

\begin{icmlauthorlist}
\icmlauthor{Diana Huang}{ed}
\icmlauthor{Shalini Keshavamurthy}{ed}
\icmlauthor{Nitin Viswanathan}{ed}
\end{icmlauthorlist}

\icmlaffiliation{ed}{Stanford University, Palo Alto, USA}

\icmlcorrespondingauthor{Diana Huang}{hxydiana@stanford.edu}
\icmlcorrespondingauthor{Shalini Keshavamurthy}{skmurthy@stanford.edu}
\icmlcorrespondingauthor{Nitin Viswanathan}{nitin.viswanathan@gmail.com}

% You may provide any keywords that you
% find helpful for describing your paper; these are used to populate
% the "keywords" metadata in the PDF but will not be shown in the document
\icmlkeywords{Machine Learning, ICML}

\vskip 0.3in
]

% this must go after the closing bracket ] following \twocolumn[ ...

% This command actually creates the footnote in the first column
% listing the affiliations and the copyright notice.
% The command takes one argument, which is text to display at the start of the footnote.
% The \icmlEqualContribution command is standard text for equal contribution.
% Remove it (just {}) if you do not need this facility.

%\printAffiliationsAndNotice{}  % leave blank if no need to mention equal contribution
\printAffiliationsAndNotice{\icmlEqualContribution} % otherwise use the standard text.

\begin{abstract}
Multi-agent scenarios frequently come up in the real world, but traditional reinforcement learning algorithms do not perform well on them due to constantly changing environments from the perspective of any one agent. We replicate multi-agent deep deterministic policy gradients (MADDPG), an algorithm tailored to multi-agent scenarios, and evaluate its performance in a 3-agent cooperative navigation OpenAI environment. Additionally, we perform many experiments to explore how changing hyperparameters affects performance and attempt to address the main weakness of MADDPG, which is that it does not scale well to larger numbers of agents.

Our results confirm that MADDPG performs significantly better than a single-agent policy gradient approach and show the importance of batch size in performance and training stability.
\end{abstract}


\section{Introduction}
\label{submission}

There are many applications where multiple agents need to learn how to act together, such as multiplayer games ~\cite{multigames}, multi-robot control ~\cite{multirobot}, communication scenarios ~\cite{communication}, and even self-play ~\cite{selfplay}.

Traditional reinforcement learning approaches focus on training a single agent and as a result they do not work well on multi-agent problems because the environment for any single agent changes over time as other agents change their policies, leading to instability during training and high variance in results ~\cite{unstable}. Additionally, experience replay as used with Q-learning cannot be directly applied in multi-agent scenarios, significantly hampering stable learning.

Research into multi-agent reinforcement learning has attempted to develop new approaches specifically for multi-agent scenarios. One such recently developed algorithm is multi-agent deep deterministic policy gradients algorithm (MADDPG), which obtains significantly improved results over other approaches by modifying DDPG to train agents while incorporating information from other agents \cite{maddpg}.

We replicate the MADDPG algorithm and apply it to an OpenAI cooperative navigation environment, where agents must learn to each navigate to a different fixed landmark without colliding. We experiment with various hyperparameters and exploration vs. exploitation methodologies to show how MADDPG's performance is affected by them. We also generalize MADDPG to handle larger numbers of agents and explore how performance changes accordingly.

Our paper has 3 key contributions:
\begin{itemize}[nosep]
\item We replicate the MADDPG algorithm and apply it to the OpenAI cooperative navigation environment
\item We experiment with various hyperparameters to show how MADDPG's performance is affected by them
\item We generalize MADDPG to handle larger numbers of agents and explore how performance changes accordingly
\end{itemize}

\section{Related Work}
\subsection{Single-agent methods}
\textbf{TODO - prior work applying PG directly, Q-learning directly, etc.}

\subsection{Multi-agent methods}
\textbf{TODO - what we've read on multi-agent approaches including MADDPG}

\section{Approach}

\subsection{Environment}

We use the cooperative navigation OpenAI Gym environment for our experiments ~\cite{openaigym}. This is a 2-D world with $N$ moving agents and $N$ fixed landmarks ($N=3$ by default), where the agents must learn to each move to a different landmark while avoiding collisions with each other.

Every agent has its own state that is an 18-dimensional vector containing its own position and velocity, relative distance to each of the landmarks, and relative distance to the other agents. The action that every agent can take is represented as a 5-dimensional vector, with 4 dimensions representing movement up/down/left/right and the last dimension representing a no-move action. This is a continuous action space, with the values for each action dimension representing acceleration in a given direction.

At each timestep, all agents are rewarded based on the distance between landmarks and agents; the closer the agents are to separate landmarks, the higher the reward. If every agent was exactly on top of a different landmark, the reward at that timestep would be 0. Agents have a non-zero size, so it is possible for them to collide in this environment. Agents receive a reward of -1 if there are collisions at any timestep. Therefore, in order to maximize reward, the agents need to each learn to move to a different landmark while also avoiding collisions with each other.

We define an episode to be 25 timesteps long, as there is no defined terminal state in this environment.

\begin{figure}
\begin{center}
\includegraphics[scale=0.5]{env-image}
\end{center}
\caption{Cooperative navigation environment. Agents must learn to each navigate to a different landmark without colliding.}
\end{figure}

\subsection{Policy Gradient}
We implement a policy gradient algorithm as a baseline. Policy gradient algorithms seek to determine the optimal policy by directly adjusting the parameters $\theta$ of a policy $\pi_\theta$ in order to maximize the objective function $J*(\theta) = \mathbb{E}_{\pi_\theta}[R]$. We also implemented a baseline and advantage for the policy gradient. Letting $\hat{A} = R_t - b(s_t)$ be the advantage estimate at time t, we can write the gradient of the policy as:
$$\nabla_\theta J(\theta) = \mathbb{E}_{s,\pi_\theta}[\nabla_\theta log \pi_\theta(a|s)\hat{A}]$$
In the above equation, the expectation is taken over many sample trajectories through the environment using $\pi_\theta$. Policy gradient algorithms are already susceptible to  high variance, and executing them in a multi-agent environments only amplifies the problem. From a single agent's perspective not only does the world change as other agents move, but the policies other agents follow change as well. Additionally, individual agents might not always get the correct gradient signal as an agent could take a good action but if other agents took bad actions, leading to overall decrease in reward, the gradient for the agent taking the correct action could still be negative.

\subsection{MADDPG}
\textbf{TODO - diagram/visualization of how MADDPG works}

We recreate MADDPG as our main algorithm for the cooperative navigation task ~\cite{maddpg}.  MADDPG takes DDPG and tailors it to multi-agent environmnnts, resulting in significantly better performance than single-agent methods. MADDPG trains separate policy networks for every agent, but uses a centralized action-value function that incorporates information across all agents to ensure a more stable learning process. 

Consider a game with $N$ agents, and let $\pi = \{\pi_1, ..., \pi_N\}$ be the associated set of agent policies that are parametrized by $\theta_1, ..., \theta_N$. Additionally let $s = \{s_1, ..., s_N\}$ represent the set of states observed by each agent. When using MADDPG to train agents, every agent additionally maintains a critic function that outputs a Q-value given the states and actions of all agents:
$$Q^\pi_i(s, a_1, ..., a_N)$$
Standard Q-learning techniques are used to update the critic function. Note that for the cooperative navigation environment that we run our experiments on, the Q-values will be the same for all agents as the reward for all agents is the same. However, this Q-function formulation where every agent learns its own $Q^\pi_i$ allows MADDPG to generalize to environments where agents have different reward functions (e.g. adversarial games).

During training, agents now have information not just about the locations of other agents, but about the actions they will take as well. Therefore, the environment is stationary for any agent as it does not depend on other agent policies:
$$P(s_i'|s_i, a_1, ..., a_N, \pi_1, ..., \pi_N) = P(s_i'|s_i, a_1, ..., a_N)$$
The policy network updates for an agent are determined by maximizing the agents centralized Q-values. The gradient update is similar to the update used in the policy gradient, except that it uses the centralized Q-value instead of only looking at the returns across trajectories:
$$\nabla_{\theta_i}J(\theta_i) = \mathbb{E}_{s, \pi_i}[\nabla_{\theta_i}log \pi_i(a_i|o_i)Q^{\pi}_i(x, a_1, ..., a_N)]$$
By incorporating the actions of all agents into the action-value function during training instead of only each individual agent, MADDPG produces better and more stable results than traditional policy gradient or Q-learning approaches. During policy execution, agents only use their learned policy networks and therefore do not use any information about other agents; MADDPG is an enhancement to the training process while leaving the execution phase unchanged.

\subsection{More scalable MADDPG}
In addition to implementing MADDPG as described in the previous section, we further modify it to scale to better support large numbers of agents. The Q function described above increases linearly in input size as more agents are added to the scenario, as it has to take every agent's observation and action as an input. In order to reduce the additional computation time needed as $N$ grows, we modify MADDPG to only look at the $K$ nearest neighbors during training, where $K < N-1$. The mathematical formulation of the algorithm remains the same, but the inputs to $Q^\pi_i$ vary based on which agents are closest to $i$ at each timestep.

\section{Experiments and Results}
\subsection{Evaluation Metrics}

In order to evaluate the performance of our algorithms, every 250 training steps we run 40 episodes using the latest learned policies and calculate various metrics over them. Our primary evaluation metric is average episode reward, which we define as the sum of agent rewards at each timestep. Episode reward is composed of both the distance between agents and landmarks as well as the number of collisions, so we look at these metrics as well.

\subsection{Experiments}

\textbf{TODO - include all of the various experiments we ran, and an analysis of each of them}

\textbf{TODO - also include a table with our main hyperparameters (copy from poster) and explain what they are for in text}

\begin{figure}
\begin{center}
\includegraphics[scale=0.4]{MADDPGvsPG}
\end{center}
\caption{Average reward achieved by policy gradient algorithm and MADDPG algorithm.}
\label{fig:MADDPGvsPG}
\end{figure}

\begin{figure}
\begin{center}
\includegraphics[scale=0.4]{MADDPGvsPG_distance}
\end{center}
\caption{Average distance from landmarks with Naive PG algorithm and MADDPG algorithm.}
\label{fig:MADDPGvsPG_distance}
\end{figure}

\textbf{REINFORCE and MADDPG.} As shown in Figure \ref{fig:MADDPGvsPG} and Figure \ref{fig:MADDPGvsPG_distance}, REINFORCE has difficulties learning a good policy as the world keeps changing from each agent's perspective. If other agents make a wrong move, it would reflect badly on this agent even though it might have picked a good action. In other words, the feedback signal is erroneous. Whereas with MADDPG, the centralized critic function takes into consideration other agents' observations and moves, so that the network could more accurately attribute rewards and penalties.  

Figures X,Y,and Z show our initial results from training with different policy network sizes. We can see that although there is a significant improvement in average reward from the start of training, the average reward is still well below 0, indicating that the agents are not able to find an optimal strategy to cooperate and cover all of the target landmarks. This result is to be expected as policy gradient will be very unstable due to the non-stationary environment from the point of view of any given agent during training. When we increased the number of layers and units per layer we saw the network reach convergence quicker, but it was still not able to improve substantially, indicating that the algorithm itself could be a bottleneck on performance as opposed to the size of the policy networks.

\textbf{Changing network size}. Finally, we experimented with reducing the size of the neural network layers for the policy and Q networks to see the impact on training. We observed that the best achievable reward in a training run with 2-layer 64-unit networks (as opposed to the default of 2 layesr with 128-unit networks) was around -175, which is noticeably lower than the -150 achieved with the default network size. This shows that the increased complexity of the neural networks do in fact help to define the right actions to take in the cooperative navigation task. 

\section{Conclusion}
We introduce the problem of multi-agent reinforcement learning and show why traditional single-agent approaches are ineffective on these problems. We implement a policy gradients as a baseline and also MADDPG, an algorithm designed for multi-agent scenarios, and apply them to the OpenAI cooperative navigation environment. 

\section{Future Work}
As future work, we would like try our approach on other multi-agent environments, in particular adversarial ones, to compare how MADDPG performs vs. policy gradients. It would also be interesting to explore how performance changes if agents are trained using different policies - e.g. in an adversarial game, what would happen if one set of agents were trained with MADDPG and a competing set were trained with policy gradients.

Additionally, we would like to explore alternative forms of generalization as we feel this is key to supporting. Instead of $Q^\pi_i$ looking at the $K$ nearest neighbors to agent $i$ for some fixed $K$, the Q-function could instead look at all neighbors within a particular distance to agent $i$. This way, the agents would have more information available to them when it is particularly important (the closer agents are to each other, the more likely it is to have a collision) while minimizing input space size when information about additional agents is less necessary.

\section*{Contributions}
Diana wrote the initial implementation of MADDPG, Shalini wrote our logging/plotting code, and Nitin wrote the initial implementation of the policy gradient baseline. All of us iterated on the code with bugfixes and enhancements and ran experiments using our implementations.

\bibliography{final}
\bibliographystyle{icml2018}

\end{document}

